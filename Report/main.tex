\documentclass[11pt, a4paper]{article}
\usepackage[margin=1.1in]{geometry}	% edit margins
\usepackage{amsmath}			% for math symbols
\usepackage{graphicx}			% including figures
\usepackage{float}			% placing figures at specific location
\usepackage{amssymb}			% math symbols
\usepackage{soul}			% for highlighting
\usepackage{mathtools}			% for text over equals sign
\newcommand\ivtequal{\stackrel{\mathclap{\normalfont\scriptsize\mbox{FVT}}}{=}}
% \usepackage{./use/mcode}		% for including code


\begin{document}

\title{Final Project: Orbit Determination\\ ASEN 5044 Statistical Estimation}
\author{Keith Cobington\\Connor Otter}
\date{December 18, 2019}
\maketitle



%%%%%%%%%%%%%%%%%%%%%%%%%%%%%%%%% INTRODUCTION %%%%%%%%%%%%%%%%%%%%%%%%%%%%%%%%%
\section{Introduction}
In order to keep track of Earth-orbiting objects, a typical observation scheme includes ground stations which measure range and range-rate to a passing satellite. 
These measurements, when combined with the known locations of the stations, allows for precises estimates of a satellites orbit state. 
In addition to an orbit estimate, it's usually useful to quantify the the uncertainty in the estimate. 
This quantity is a based on the uncertainty in the motion of the satellite in orbit, and the uncertainty in the incoming measurements. 
Neither of these uncertainties are necessarily known.
In order to accurately predict an estimate uncertainty, both of these uncertainties must be balanced with each other. 
This is frequently carried out with predictor-corrector estimation algorithm such as the Linearized Kalman Filter (LKF) or Extended Kalman Filter (EKF.)

In this report, we explore the performance of the LKF and EKF on nonlinear systems with dynamic uncertainty (process noise) and measurement uncertainty (measurement nose.) 
In developing these algorithms, we assume the process and measurement noise are unknown.
We then iterate on different process noise and measurement noise covariance matrices and evaluate the algorithms using \_\_\_\_ (NEES) and \_\_\_\_ (NIS) tests until suitable values for uncertainty are found. 


% \subsection{Physical System}
\begin{figure}[H]
	\centering
	\includegraphics[width=.4\textwidth]{./Figures/system_setup.png}
	\caption{Conceptual visualization of system.}
	\label{fig: system}
\end{figure}


\subsection{Part I. Basic System Analysis}

\subsection{CT Model Jacobians}

\subsection{DT Linearized Model Matrices}

\subsection{Simulated Nonlinear vs. Linearized System}
\subsubsection{Nonlinear perturbed trajectory simulation vs. linear perturbation propagation}
\subsubsection{Nonlinear perturbed trajectory measurement simulation vs linear perturbation mapping}


%%%%%%%%%%%%%%%%%%%%%%%%%%%%%%%%% CONCLUSION %%%%%%%%%%%%%%%%%%%%%%%%%%%%%%%%%
\section{Conclusion}




%%%%%%%%%%%%%%%%%%%%%%%%%%%%%%%%% APPENDIX %%%%%%%%%%%%%%%%%%%%%%%%%%%%%%%%%
\newpage
\section*{Appendix}
Attached here is the code used for the numerical analysis and plotting entailed in this project.
%\lstinputlisting{../main.m}


\end{document}


